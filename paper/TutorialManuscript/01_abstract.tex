\begin{abstract}
%OLD:
The weighted ensemble (WE) strategy has been demonstrated to be highly efficient in generating pathways and rate constants for rare events such as protein folding and protein binding using atomistic molecular dynamics simulations. 
Here we present two sets of tutorials instructing users in the best practices for preparing, carrying out, and analyzing WE simulations for various applications using the WESTPA software. 
%Users are expected to already have significant experience with running standard molecular dynamics simulations using the underlying dynamics engine of interest (e.g. Amber, Gromacs, OpenMM). 
The first set of more basic tutorials describes a range of simulation types, from a molecular association process in explicit solvent to more complex processes such as host-guest association, peptide conformational sampling, and protein folding. 
%
%NEW: 
The second set ecompasses six advanced tutorials instructing users in the best practices of using key new features and plugins/extensions of the WESTPA 2.0 software package, which consists of major upgrades for larger systems and/or slower processes. 
The advanced tutorials demonstrate the use of the following key features: (i) a generalized resampler module for the creation of “binless” schemes, (ii) a minimal adaptive binning scheme for more efficient surmounting of free energy barriers, (iii) streamlined handling of large simulation datasets using an HDF5 framework, (iv) two different schemes for more efficient rate-constant estimation, (v) a Python API for simplified analysis of WE simulations, and (vi) plugins/extensions for Markovian Weighted Ensemble Milestoning and WE rule-based modeling for systems biology models.
Applications of the advanced tutorials include atomistic and non-spatial models, and consist of complex processes such as protein folding and the membrane permeability of a drug-like molecule. 
Users are expected to already have significant experience with running conventional molecular dynamics or systems biology simulations.% and completed the previous suite of WESTPA tutorials. 
\end{abstract}